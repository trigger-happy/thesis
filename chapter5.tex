\chapter{RESULTS}

\section{Code Structure}
The code for the game was structured such that it fulfilled the necessary
requirements of being able to run both on the CPU and the GPU without requiring
too many modifications. The first major hurdle was the way the memory had to be
structured to maximize performance on the GPU. The concept of a structure of
arrays was used to fulfill this need. Instead of declaring a tank class and
then having several instances of this class stored in an array to represent
each tank in the game, the tank class was declared such that each of its
parameters is an array. A tank instance is then identified by an array index
which is then used to get and set its properties from the arrays. This ensured
that whenever the program had to access the x position of the tanks, the memory
access would be coalesced into a single read from a block of memory, thereby
improving performance. Nearly all the objects in the game were structured as
such and an unsigned integer was used to identify any particular instance of an
object.

The next problem was the memory transfer between the CPU and GPU. The problem
was that it was necessary to check if the terminating conditions have been
fulfilled for the fitness checking stage of the evolution, but transfering
memory between the CPU and the GPU (or more specifically the GPU memory to CPU
RAM) is a time consuming operation. To solve this, memory trasfers from the GPU
memory to CPU RAM is only done every 30 frames. This minimized the number of
memory transfers that had to be done while still being able to check
periodically for the terminating conditions of the fitness stage. It should be
noted however that the selection of this number of frames was done arbitrarily
and still open to experimentation for improving the performance.

\section{Tabular Results} 
We have set the maximum amount of generations to 24, as we have observed 
that the AI often does not evolve any further after the 24th generation. The 
table \ref{table:CPU vs GPU total runtime} and 
\ref{table:CPU vs GPU total runtime part 2} contains the total time in each
run per seed for both CPU and GPU. The ANOVA test indicates that the result
we got was not a type 1 error as shown in figure \ref{fig:anova_result}.
\begin{figure}
	\centering
		\graphicspath{{images/}}
		\includegraphics[width=260 pt]{Anova_Results.png}
	\caption{Anova Results}
	\label{fig:anova_result}
\end{figure}
Thus, the GPU can be considered to be faster than the CPU as shown in figure
\ref{fig:anova_result_graph}.
\begin{figure}
	\centering
		\graphicspath{{images/}}
		\includegraphics[width=260 pt]{Anova_Result_Graph.png}
	\caption{CPU vs GPU Graph}
	\label{fig:anova_result_graph}
\end{figure}

\begin{table}
\caption{CPU vs GPU Total Runtime Seeds 0 and 1}
\centering
 \begin{tabular}{ | l | l | l |}
    \hline
    Time & Seed & Processor \\ \hline
    471.94 & 0 & CPU \\ \hline
    469.5 & 0 & CPU \\ \hline
    468.12 & 0 & CPU \\ \hline
    468 & 0 & CPU \\ \hline
    478.42 & 0 & CPU \\ \hline
    467.59 & 0 & CPU \\ \hline
    465.22 & 0 & CPU \\ \hline
    465.5 & 0 & CPU \\ \hline
    467.64 & 0 & CPU \\ \hline
    467.19 & 0 & CPU \\ \hline
    85 & 0 & GPU \\ \hline
    83.96 & 0 & GPU \\ \hline
    84.07 & 0 & GPU \\ \hline
    83.91 & 0 & GPU \\ \hline
    83.99 & 0 & GPU \\ \hline
    83.94 & 0 & GPU \\ \hline
    83.95 & 0 & GPU \\ \hline
    83.96 & 0 & GPU \\ \hline
    83.93 & 0 & GPU \\ \hline
    83.93 & 0 & GPU \\ \hline
    469.11 & 1 & CPU \\ \hline
    466.16 & 1 & CPU \\ \hline
    466.18 & 1 & CPU \\ \hline
    468.07 & 1 & CPU \\ \hline
    465.86 & 1 & CPU \\ \hline
    466.64 & 1 & CPU \\ \hline
    467.87 & 1 & CPU \\ \hline
    468.76 & 1 & CPU \\ \hline
    473.68 & 1 & CPU \\ \hline
    501.45 & 1 & CPU \\ \hline
    83.97 & 1 & GPU \\ \hline
    84.79 & 1 & GPU \\ \hline
    83.78 & 1 & GPU \\ \hline
    83.83 & 1 & GPU \\ \hline
    83.87 & 1 & GPU \\ \hline
    83.85 & 1 & GPU \\ \hline
    83.88 & 1 & GPU \\ \hline
    83.8 & 1 & GPU \\ \hline
    86.9 & 1 & GPU \\ \hline
    86.91 & 1 & GPU \\ \hline
    \end{tabular}
\label{table:CPU vs GPU total runtime}
\end{table}

\begin{table}
\caption{CPU vs GPU Total Runtime Seeds 2 and 3}
\centering
 \begin{tabular}{ | l | l | l |}
    \hline
    Time & Seed & Processor \\ \hline
    497.14 & 2 & CPU \\ \hline
    462.1 & 2 & CPU \\ \hline
    456.83 & 2 & CPU \\ \hline
    489.04 & 2 & CPU \\ \hline
    472.8 & 2 & CPU \\ \hline
    471.66 & 2 & CPU \\ \hline
    473.58 & 2 & CPU \\ \hline
    474.54 & 2 & CPU \\ \hline
    474.99 & 2 & CPU \\ \hline
    474.22 & 2 & CPU \\ \hline
    86.13 & 2 & GPU \\ \hline
    90.46 & 2 & GPU \\ \hline
    90.48 & 2 & GPU \\ \hline
    85.75 & 2 & GPU \\ \hline
    84.77 & 2 & GPU \\ \hline
    84.96 & 2 & GPU \\ \hline
    85.06 & 2 & GPU \\ \hline
    84.92 & 2 & GPU \\ \hline
    84.96 & 2 & GPU \\ \hline
    86.18 & 2 & GPU \\ \hline
    461.94 & 3 & CPU \\ \hline
    464.13 & 3 & CPU \\ \hline
    464.6 & 3 & CPU \\ \hline
    461.13 & 3 & CPU \\ \hline
    464.12 & 3 & CPU \\ \hline
    463.9 & 3 & CPU \\ \hline
    463.06 & 3 & CPU \\ \hline
    464.25 & 3 & CPU \\ \hline
    463.54 & 3 & CPU \\ \hline
    464.36 & 3 & CPU \\ \hline
    84.22 & 3 & GPU \\ \hline
    84.27 & 3 & GPU \\ \hline
    84.24 & 3 & GPU \\ \hline
    84.24 & 3 & GPU \\ \hline
    84.27 & 3 & GPU \\ \hline
    84.25 & 3 & GPU \\ \hline
    84.2 & 3 & GPU \\ \hline
    84.23 & 3 & GPU \\ \hline
    84.28 & 3 & GPU \\ \hline
    84.22 & 3 & GPU \\ \hline
    \end{tabular}
\label{table:CPU vs GPU total runtime part 2}
\end{table}


