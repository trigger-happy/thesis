\chapter{METHODOLOGY}
% Edit this to ensure that it turns into a clear step by step process
% for replicating the research
%
% Specify the software development cycle...
% also indicate the tools used to program (hardware and software)
%
% Instruments
% Procedures
% Data Analysis (this is where statistics come in) <-- DO NOT FORGET
\section{Implementation Details}

\subsection{Application and Tools}
The application to be developed for optimizing the AI will have 2 versions.
The first version will run the game and perform the genetic operations entirely
on the CPU. This version of the application will serve as a benchmark on the
time it takes for an optimized solution to be found through genetic algorithms.
The other version will run the game simulation on the GPU and the rest of the
code in the CPU.


The following are the tools, libraries and systems used for developing and testing
the application:

\begin{enumerate}
  \item Fedora 13 Linux (for the operating system)
  \item gcc 4.4.4 compiler suite
  \item Boost 1.43.0 libraries
  \item Nvidia CUDA 3.1 SDK
  \item CMake 2.8 build system
  \item ClanLib 2.1 game engine (to be used for viewing the AI's performance)
\end{enumerate}

The hardware used for development is any Intel based multicore CPU and any Nvidia
graphics card with CUDA compute capability greater than 1.0. The main system to be
used for gathering results is a Fedora 13 desktop with an Intel Core 2 Quad for the
CPU and an Nvidia GTS 250 for the graphics card.


\subsection{Simulation Details}
The game that we intend to use as a test bed for our AI is a simple 2D tank game.
In order to simplify the process, only a subset of the game will be implemented.
The game will have a single tank controlled by the AI and the objective of the AI
is to stay alive for as long as possible by evading all the bullets that approach
it from random points around the playing field. The AI will take note of three
elements in evading the bullets in the playing field.

\begin{enumerate}
 \item Collision State - a state where the AI will take note of the speed 
of objects in the field.
 \item Direction State - a state where the AI will take note 
of the direction it's facing.
 \item Distance State - a state where the AI will
take note of the distance of the tank and the bullets.
\end{enumerate}

In this scenario, the AI will try to dodge the bullets that will collide with
it. The fitness of the AI will be determined by the duration between the start of
the game and the moment the tank collides with a bullet.


In a single generation, there will be a large number of individual genomes. The number
can range from 1,024 to 4,096 (exact value to be determined through experimentation).
Each individual genome will go through the evasion test bed to obtain its fitness score.
When the simulation time has passed a time limit or the tank in each simulation has been
shot, the test for that particular generation will end and the individuals are ranked
according to their fitness score. The top 15\% of individuals will automatically become
part of the next generation. The remaining 85\% will become the parents of the next generation 
through reproduction. The test will then repeat the process and evaluate the new
generation of individuals. The entire process will end only when the fitness score
of the best individuals in succeeding generations no longer increase after multiple iterations.


\subsection{Data Analysis}
The success of the GPU based application's performance over the CPU based version is measured
by the time it takes for the GPU version to find an optimal solution for the AI's evasion
behavior over the CPU version. Both applications will be run a number of times and their
average running time will be used for comparison. Because of the random nature of genetic
algorithms, the optimal solution found in each run may not necessarily be the same. We will
therefore consider a solution as optimal if it performs just as well as other candidate
solutions when placed in identical levels. Performance is measured by how long the AI
can survive without getting hit by any bullet. Two levels are identical when they have the
same number of bullets fired and every bullet fired in one stage has the same starting position,
firing time and trajectory as its corresponding bullet in another stage. With this, highly
similar solutions should yield AI players with very similar lifespans.