\chapter{METHODOLOGY}
% Edit this to ensure that it turns into a clear step by step process
% for replicating the research
%
% Specify the software development cycle...
% also indicate the tools used to program (hardware and software)
%
% Instruments
% Procedures
% Data Analysis (this is where statistics come in) <-- DO NOT FORGET
\section{Implementation Details}
We used a 2D tank game as a test bed for optimizing the game AI. In
this game, there would be several tanks which are controlled by the
AI. The objective of the game is to eliminate the other tanks and be
the last tank remaining.  Tanks are able to move and attack to accomplish
the objective. The best AI should ideally be able to shoot accurately
and dodge effectively in order to eliminate the other tanks and
survive as long as possible. 

\subsection{Simulation}
The behavior we want to optimize is the evasion behavior.  The test bed for the 
behavior would involve a single tank in the center of the field and random points 
in the field where bullets will travel towards the tank.  The AI will take note of
three elements in evading. 

\begin{enumerate}
 \item Collision State - a state where the AI will take note of the speed 
of objects in the field.
 \item Direction State - a state where the AI will take note 
of the direction it's facing.
 \item Distance State - a state where the AI will
take note of the distance of the tank and the bullets.
\end{enumerate}

In this scenario, the AI will try to dodge the bullets that will collide with
it.  The length of time the AI managed to survive will serve as the fitness score
for this particular individual.


In a single generation, there will be a large number of individual genomes. The number
can range from 1,024 to 4,096. Each individual genome will go through the evasion
test bed to obtain its fitness score.  When the simulation time has passed a time 
limit or the tank in each simulations has been shot, the test for that 
particular generation will end and the individuals are ranked according to their fitness
score. The top 15 percent of individuals will automatically become part of the next
generation. The remaining 85 percent will become the parents of the next generation 
through reproduction. The test will then repeat evaluating the individuals of the new
generation. The entire process will end only when the fitness score of the best individuals
in succeeding generations no longer increase after multiple iterations.

\subsection{Application}
Two version of the application will be created. The first one will be CPU-based version of
the application. This version will be optimized for the CPU and will make use of threads
in multi-core CPUs. The other application will be GPU-based. We will optimize it to make use
of Nvdia's Compute Unified Device Architecture. 


The CPU version will run a single test bed for scoring an individual. The more cores the CPU has
the more threads that will be utilized to speed up the process. On the other hand, the GPU version
will assign one thread block per test bed. Each thread in a block will be responsible for 
updating a specific object in a particular instance. A single thread will be responsible
for executing the AI functions. After evaluating a generation, the ranking and the genetic 
operations will be handled by the CPU.

\subsection{Genetic Design}


\subsection{Measuring the Results}