\chapter{METHODOLOGY}
% Edit this to ensure that it turns into a clear step by step process
% for replicating the research
%
% Specify the software development cycle...
% also indicate the tools used to program (hardware and software)
%
% Instruments
% Procedures
% Data Analysis (this is where statistics come in) <-- DO NOT FORGET
\section{Implementation Details}

Our first set of research questions involve the design of the application itself.
There are several things that must be done during the design phase. The first is
the design of the actual game where the AI must be placed in. For this we chose to
create a simple 2D tank game where the objective is to shoot the other tanks
while at the same time avoiding destruction from the projectiles of other tanks. We
chose this kind of game as it is simple enough to create within a short amount of time
while at the same time being complex enough for the AI to perform a wide variety of
actions. Additional game elements like health points, player score, power ups and
obstacles may be implemented to increase or decrase the complexity of the game.


The next design element that needs to be tackled is the implementation of the AI itself.
Although we will be using Genetic Algorithms to evolve the AI in becoming optimal, it
will still require a framework which it will operate in. Such implementations include
Finite State Machines, Fuzzy State Machines, Hierarchical State Machines, Behavior Trees
and Hierarchical Task Networks. Each of these has its pros and cons and the choice of
which AI implementation to use will determine how the AI's "genetic" structure will be
designed and evolved.


Once we have chosen which AI implementation to use, the next design issue that needs to
be addressed is the way the "genes" should be structured for evolution. The simplest
structure would be a vector of numeric values that can be used to tune some specific
parameters in the AI's code. Things like the minimum amount of health points that the agent can have
before it attempts to withdraw from battle to repair or the minimum distance between the
agent and an incoming projectile before the agent should attempt to evade. This structure
implies that the AI's decision making is already fixed and the only things that change
are the various parameters that will trigger a yes or no in any single decision. There
are other possible genetic structures, but these may be too complicated to implement
within a reasonable time frame.