\chapter{METHODOLOGY}
% Edit this to ensure that it turns into a clear step by step process
% for replicating the research
%
% Specify the software development cycle...
% also indicate the tools used to program (hardware and software)
%
% Instruments
% Procedures
% Data Analysis (this is where statistics come in) <-- DO NOT FORGET
\section{Implementation Details}

\subsection{Application and Tools}
The application to be developed for optimizing the AI will have 2 versions.
The first version will run the game and perform the genetic operations entirely
on the CPU. This version of the application will serve as a benchmark on the
time it takes for an optimized solution to be found through genetic algorithms.
The other version will run the game simulation on the GPU and the rest of the
code in the CPU.


The following are the tools, libraries and systems used for developing and testing
the application:

\begin{enumerate}
  \item Fedora 13 Linux (for the operating system)
  \item gcc 4.4.4 compiler suite
  \item Boost 1.43.0 libraries
  \item Nvidia CUDA 3.1 SDK
  \item CMake 2.8 build system
  \item ClanLib 2.1 game engine (to be used for viewing the AI's performance)
\end{enumerate}

% We may have to use the new research comp for testing instead.
The hardware used for development can be any Intel based multicore CPU and any Nvidia
graphics card with CUDA compute capability greater than 1.0. The main system to be
used for gathering results is a Fedora 13 desktop with an Intel Core 2 Quad for the
CPU and an Nvidia GTS 250 for the graphics card. The main system to be utilized for
developing and debugging the application is a Fedora 13 laptop with an Intel
Core i3 for the CPU and a Nvidia Geforce 310M for the GPU.


\subsection{Simulation Details}
The game that we intend to use as a test bed for our AI is a simple 2D tank game.
In order to simplify the process, only a subset of the game will be implemented.
The game will have a single tank controlled by the AI and the objective of the AI
is to stay alive for as long as possible by evading all the bullets that approach
it from fixed points near the edges of the field. The firing sequence and the rate
of fire is determined at the start of the program and remain constant throughout
the testing procedure. The AI will take note of three elements in evading the
bullets in the playing field.

\begin{enumerate}
 \item Collision State - a state where the AI will take note of the speed 
of objects in the field.
 \item Direction State - a state where the AI will take note 
of the direction it's facing.
 \item Distance State - a state where the AI will
take note of the distance of the tank and the bullets.
\end{enumerate}

In this scenario, the AI will try to dodge the bullets that will collide with
it. The fitness of the AI will be determined by the duration between the start of
the game and the moment the tank collides with a bullet.


In a single generation, there will be a large number of individual genomes. The number
can range from 1,024 to 4,096 (exact value to be determined through experimentation).
Each individual genome will go through the evasion test bed to obtain its fitness score.
When the tank in each simulation has been
shot, the test for that particular generation will end and the individuals are ranked
according to their fitness score. The score of an individual is assigned a value
between 0 to 1 inclusive. A value of 0 would mean that the AI was shot at the
very moment that the game has started while a score of 1 would mean that the AI
was able to survive until the end of the time limit.
Due to the nature of the game, there is a possibility
that the AI will remain unharmed for an indefinite amount of time. To avoid this,
we will be capping the game time to a maximum of 5 minutes. Individuals that are
still alive by then are given the score of 1.0 and may be considered optimal for
that particular stage configuration.
The top 15\% of individuals will automatically become
part of the next generation. The remaining 85\% will become the parents of the next generation 
through reproduction. The test will then repeat the process and evaluate the new
generation of individuals. The evolution process will end only when the fitness score
of the best individuals in succeeding generations no longer increase after multiple iterations
or when an individual with a fitness score of 1.0 is found during the evolution
process.


\subsection{Data Analysis}
Data will be gathered by running the CPU and GPU based applications a number of
times. There are 3 kinds of data to be gathered in each run of the application.

\begin{enumerate}
  \item The fitness score of the best individual.
  \item The generation in which the individual was found.
  \item The time it took to find that individual (in physical time).
\end{enumerate}

Data will be laid out in a table where the columns indicate the range of fitness
values that are fixed into intervals of .30 while each row represents a single
generation. This results in the table having columns representing the fitness
ranges 0-0.33, 0.34-0.66, and so on. The data stored in this table
will be the time the application took in seconds (in physical time) to obtain
the best individual. Multiple results from the same application that end up in
the same cell will have their mean value stored. Comparison is done only in the
cells which have a recorded runtime from both the GPU and CPU based application.
Cells with only one or the other are ignored. Scoring is done by tallying the
number of cells where the runtime of one application is faster than the other.
The application with the higher score can be concluded as the faster 
implementation.