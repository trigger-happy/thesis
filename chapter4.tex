\chapter{METHODOLOGY}
% Edit this to ensure that it turns into a clear step by step process
% for replicating the research
%
% Specify the software development cycle...
% also indicate the tools used to program (hardware and software)
%
% Instruments
% Procedures
% Data Analysis (this is where statistics come in) <-- DO NOT FORGET
\section{Implementation Details}

\subsection{Application and Tools}
We have developed an evolver application which optimized the gene and a runner
application which presented the gene. Furthermore, we also have developed a CPU and
a GPU version for both applications. The first version ran the game and performed 
the genetic operations entirely on the CPU. This version of the application served 
as a benchmark on the time it took for an optimized solution to be found through 
genetic algorithms. The other version ran the game simulation on the GPU and 
the rest of the code in the CPU.


The following were the tools, libraries and systems used for developing and testing
the application:

\begin{enumerate}
  \item Fedora 14 Linux (for the operating system)
  \item gcc 4.4.4 compiler suite
  \item Boost 1.43.0 libraries
  \item Nvidia CUDA 3.2 SDK
  \item CMake 2.8 build system
  \item ClanLib 2.1 game engine (to be used for viewing the AI's performance)
\end{enumerate}

% We may have to use the new research comp for testing instead.
The main system used for gathering results was a Fedora 13 desktop with an 
Intel(R) Core(TM)2 Quad CPU Q9550 @ 2.83GHz for the CPU and an 
Inno3d GeForce GTX 470 (Hawk edition) for the graphics card. The CPU used 8
threads to compute for the result while the GPU used 1024 thread block with
32 threads per block to compute. The main system utilized for developing and 
debugging the application was a Fedora 13 laptop with an Intel Core i3 for 
the CPU and a Nvidia Geforce 310M for the GPU.


\subsection{Evolution Method}
The game that we used as a test bed for our AI was a simple 2D tank game.
In order to simplify the process, only a subset of the game was implemented.
The game had a single tank controlled by the AI and the objective of the AI
was to stay alive for each scenario by evading all the bullets that approached
it from fixed points on the field. The AI took note of five elements in evading the
bullets in the playing field.

\begin{enumerate}
 \item Bullet Heading - the direction of where the bullet is heading.
 \item Bullet Position - the position of the bullet in the field.
 \item Distance State - a state where the AI takes note of the distance 
of the tank and the bullets.
 \item AI Thrust - how the AI will move (move forward, move backward, or stop).
 \item AI Heading - the indented direction the AI will move among the 18
possible direction.
\end{enumerate}

We used 1024 individuals for our tank AI. An individual was tested using 54 different
scenarios. We used 18 different angles the bullets can shoot from. We also used 4
distance states. However, the bullet fired at the tank at distance state 0 is nearly
impossible to dodge in nearly all cases. Therefore, we used the last 3 distance states.
The evading AI will then check for the bullet status, which includes the position, heading,
and distance, every 30 frames. The AI will recieve 1 point for every scenario it survives.
Finally, a scenario only ends when either of the 2 conditions are met:

\begin{enumerate}
 \item Evading tank gets shot.
 \item Fired bullet reaches its maximum distance and dies out.
\end{enumerate}

Genetic evolution was performed after each generation. The top 15\% of individuals
were automatically included as part of the next generation. The remaining 85\% became
the parents of the next generation through reproduction. We got two individuals and
picked a random point in which to split the individuals. The first part of the 
first individual was then paired with the second part of the second individual to 
create a new individual. Furthermore, the second part of the first individual was
paired with the first part of the second individual to create another individual.
Lastly, each of the new individual has a 50\% chance of mutating, i.e., a genome
in the individual will have a different value. The process is repeated for the
rest of the individuals in the lower 85\% bracket.


We used 4 different initial seeds. We ran each seed 10 times for both the CPU and 
the GPU. Furthermore, the CPU and GPU are run separately, not simultaneously. This
is due to the GPU using CPU cycles during the reproduction phase.

\subsection{Data Analysis}
Data was gathered by running the CPU and GPU based applications a number of
times. There are 4 kinds of data that were gathered in each run of the application.

\begin{enumerate}
  \item The number of scenarios the best individual in that generation manages 
to survive.
  \item The fitness score of the best individual.
  \item The generation number.
  \item The average time it took to finish all 54 scenarios for all individual of 
that generation (in physical time).
\end{enumerate}

The number of survived scenarios, fitness score, and the generation number were used for 
comparision between the CPU and GPU of the same seed so as to know the consistency
of the AI result. The time was used to compare for the speed at which the CPU and GPU
completes the evolution. For each generation of the CPU, we got the average evolution
time of each generation. We then averaged the time from all same generation within a seed.
We repeated the process for all generation within a seed. After that, we averaged the 
total evolution time of a seed. We then repeated the whole process for the GPU. After
which, we compare the average evolution time.