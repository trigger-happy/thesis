\subsubsection*{Significance of the Study}
Genetic Algorithms have applications over a wide variety of fields.
Inspired by evolutionary biology, it can find the optimal solution
for a given problem.  Because of this, some real world applications
have been made.  For example, we can make more sophisticated AI using 
Genetic Algorithms.  These AIs will be smarter in the sense that it can ``learn''.  
This can be very invaluable in a game due to making the AI a lot more challenging.  
With the age of graphics reaching its peak, the need for better AI starts being an
ingredient for a succesful commercial game\cite{Yue06}.  Because of inherent ability
of Genetic Algorithms in learning, AIs can be made more robust.  They could 
display behaviors that were not seen in the previous playthrough.  Thus, 
this could make more challenging AIs after each playthrough.  This was made an 
example of through Megaman 2 in the boss fight against Air Man.  It kept trying different 
set of inputs to defeat Air Man.  Each generation performed better than the last 
until it reached Generation 13 whereby its fitness score didn't increase anymore\cite{website:Kuliniewicz09}.  


Other application of Genetic Algorithm can be found in companies like First Quadrant Corp.
It is an investment management firm, used Genetic Algorithms to model yield on 
investment funds\cite{website:Davis}.  Through this method, they are have a ``significant improvement''
of over USD 128 Billion under investment\cite{website:Davis}.  Furthermore, in the field of Telecommunications, 
Cox Associates is using Genetic Algorithms for their cellular clients\cite{website:Davis}.  
Due to this, they are saving millions of dollars.  Another application can be found 
in the field of traffic control, more specifically in traffic lights and pedestrian crossing
control\cite{Turky09}.  

It is very obvious that a lot of studies have been done in the fields of Genetic Algorithm.
However, there aren't (that we know of) any studies done in improving Genetic Algorithms
using the GPU for Game AIs.  It has been proven that Genetic Algorithms are limited without
the processing power of the GPU\cite{Banzhaf09}.  Therefore, the processing power of the GPU can give better 
results because of its parallel structure.  Thus, it is possible to have a large population size.
With this, the AI can evolve and find better results , i.e., fitness scores, creating behaviors that are normally
unseen in normal situation.  It is here that we belive that we 
could make a contribution to the field of both Genetic Programming and Game Development (AI development).