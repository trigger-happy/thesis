\chapter{REVIEW OF RELATED LITERATURE}
\section{Data Visualization}
By interacting with the physical world, human beings have developed a more coordinated eye-brain system that allows us to easily interpret information from images \cite{wolff}. Because of this unique ability, visualization has been the most effective way of conveying scientific data, which is complex and abstract in its raw form.

\bigskip
Visual representation of data may be classified into two general forms: information visualization and scientific visualization, the broadest of which is information visualization. It concerns the visual representation of a collection of non-numerical data, like articles, files, databases and networks \cite{friendly}. The second classification is scientific visualization which is more focused on modeling of natural phenomena through complex data structures \cite{wolff}. Scientific visualization is not only limited to modeling. It may also represent scientific data through other forms of visualization techniques.

\bigskip
With the unimaginable advancement in computing, data visualization evolved from simple point plots and bar graphs to complex yet aesthetically-appealing models. Several modern data visualization applications do not only represent data but also engage human imagination and creativity. To create these applications, programmers, statisticians and researchers follow a sequence of steps to visually represent the data patterns or trends that they wish to highlight.

\section{Steps for Visualizing Data}
The process of visualizing data begins with data acquisition \cite{fry}. In this process, data may be acquired from a file on a storage device, such as hard disk, or a source on a network. This process can be either extremely complicated in which the system has to gather only the useful data in a large system or very simple in which the system only has to read from a text file. Acquisition not only involves how data is obtained but also how the user can download the data. This will greatly affect how the system will be designed.  For example, if the system is accessible online, then the time it will take to download the data to the user's browser must be taken into account. Since the data is usually a part of a much bigger data set on the server, the data must be structured in such a way that common subsets can easily be retrieved.  

\bigskip
The next step is parsing \cite{fry}. Once the data file is acquired, each line must be broken into individual parts which will be tagged with its intended use. Then, conversions will be done to each field to change it into useful format. For example, data values passed to a Java program need to be converted to Java-specific data types, specifically, String, Float, Integer and Character. 

\bigskip
After this, the data is filtered to remove portions that will not be used \cite{fry}. This step also includes mathematical computation to deduce data that are not provided by the source and normalization to standardize the data values. This step is followed by mining or the analysis stage. To deduce patterns or mathematical equations that represent the data, statistics and data mining are used in this stage.

\bigskip
After mining, the data is represented through a basic visual model \cite{fry}. The model that will be used depends on the visualization technique best fitted for the acquired data and the data pattern that will be highlighted. Although this is the third to the last stage, this greatly influences data acquisition and filtering. After this stage, the resulting visualization is further refined to put more emphasis on particular data or to increase readability by changing some attributes, such as color. Finally, user interaction is added to the application allowing the user to manipulate and explore the data either by limiting the data values shown on the visualization or changing the perspective.

\bigskip
In order to come up with a visualization application, these series of steps must be followed. However, the most crucial stage of all of these steps is visual representation which can only be achieved by employing one or a combination of visualization techniques. 

\section{Visualization Techniques}
The two most basic visualization techniques are point plots and bar graphs \cite{wolff}. Despite the simplicity of these graphs, these are the more widely used means of presenting data. From these graphs, researchers can easily deduce the correlation between two variables, which is a fundamental question in a scientific inquiry. However, bar graphs may sometimes obscure the data rather than make it clearer. Exponential growth, for example, is more obvious when the data is graphed using connected point plots rather than bar graphs. On the other hand, the difference in the values is much clearer in stacked bar graphs than in scattered plots. The choice of technique in visualizing data, therefore, depends on the type of data set and parameters that will be shown.

\bigskip
A higher form of visualization uses contours and images \cite{wolff}. Unlike the first one, two parameters are considered rather than one. Examples of which are applications that show heat maps.

\bigskip
A much more advanced technique uses vectors \cite{wolff}. This is used when the data set has both magnitude and direction so that the output is a vector representation. An example of which is a visualization of wind velocity.    

\bigskip
Multiparameter scalar and vector data sets demand a much more sophisticated approach to visualization \cite{wolff}. Often these data sets are represented using colored 3D mapping of data with the rest of the parameters mapped on top of it.

\bigskip
Aside from the standard 2D/3D visualization techniques, there are other visualization techniques that combine basic visualization principles \cite{keim}. Examples of which are geometrically transformed displays, iconic displays and dense pixel displays.

\bigskip
Geometrically transformed display techniques (see Figure \ref{fig:parallel}) are used to highlight significant changes in multidimensional data sets \cite{keim}. Examples of techniques under this class are Prosection Views, Hyperslice and Parallel Coordinates technique. The idea behind parallel coordinates technique is to visualize each of the data item as a polygonal line that intersects each of the vertical display axes at the data value of the considered attribute. 

\bigskip
\begin{figure}
	\centering
		\includegraphics{parallel.png}
	\caption{Parallel Coordinate Visualization \cite{keim}}
	\label{fig:parallel}
\end{figure}

\bigskip
In iconic displays, visualization is done by mapping the attribute values of a multidimensional data item to the features of an arbitrary icon, such as star, stick figure, and little faces \cite{keim}.  In the stick figure technique, two attributes determine the density of the head and body while the remaining attributes determine the other features of the stick figure, specifically the angles and/or limb length.

\bigskip
\begin{figure}
	\centering
		\includegraphics{stick.png}
	\caption{Stick Figure Visualization \cite{alfaruque}}
	\label{fig:stick}
\end{figure}

\bigskip
Another group of visualization techniques is the dense pixel technique \cite{keim}. In this technique, colored pixels are used to represent each dimension value and pixels representing the same dimension are grouped into adjacent areas. The pixels are arranged on the screen depending on the purpose of the visualization. When pixels are arranged using the most appropriate technique, valuable information such as local correlations, dependencies and hot spots can be easily deduced from the resulting visualization. 

\bigskip
Two of the most common means of arranging the pixels are recursive pattern technique and circle segments technique \cite{keim}.  In recursive pattern technique (see Figure \ref{fig:recursive}), the pixels are arranged through generic recursion. This technique is applied to datasets that are ordered based on one attribute (e.g. time series data). The user may control the arrangement of the pixels by specifying parameters in each recursion level, creating more informative substructures. For each recursion, the pixels are arranged within a rectangle, the height and weight of which is specified by the user. First, the pixels are arranged from left to right based on the parameter specified by the user, then from right to left at the bottom then left to right again, and so on. The same arrangement is done on the succeeding recursion levels except that the elements arranged in level \textit{i} is the result of the arrangement done on the previous level, level (\textit{i}-1).

\bigskip
\begin{figure}
	\centering
		\includegraphics{recursive.png}
	\caption{Dense Pixel Displays: Recursive Pattern Technique\cite{keim}}
	\label{fig:recursive}
\end{figure}

\bigskip
In the circle segments technique (see Figure \ref{fig:circle}), each single colored pixel corresponding to one attribute value is arranged in a segmented circle \cite{keim}. Each segment represents one dimension and pixels representing the same dimension are grouped into one segment. The pixels are plotted by drawing a line perpendicular to the segment and arranged from the center to the outside. This produces visualization with the difference between the attribute values emphasized at the center.

\bigskip
\begin{figure}
	\centering
		\includegraphics{circle.png}
	\caption{Dense Pixel Displays: Circle Segments Technique\cite{keim}}
	\label{fig:circle}
\end{figure}

\bigskip
The developments in visualization techniques have made visual representations more informative yet easier to interpret. Because of this, data visualization applications have been used in several fields outside of research, such as weather forecasting. 

\section{Weather Visualization}
Visualization applications that handle weather information fall under a more specific branch of data visualization called weather visualization \cite{lodha}. An example of weather visualization tool is Aviation Weather Data Visualization Environment (AWE). It is tailor-fitted for general aviation pilots. It shows a real-time graphical representation of weather information on a cartographic grid. 

\bigskip
In spite of the diversity of the application of weather forecasting, developers have failed to take advantage of this to further develop their weather visualization systems. Thus, most of the weather visualization tools have not been developed to cater to more specific needs, such as disaster prediction and management. However, there are but a few studies conducted on weather visualization that was aimed at developing tools that are not only limited to weather forecasting. 

\bigskip
A study conducted in Hong Kong tapped into the versatility of weather data visualization tools to analyze air pollution \cite{chan}. The developers of the system integrated three visualization techniques, namely, polar system, parallel coordinates and weighted complete graph. 

\bigskip
Although the two aforementioned studies were successful, the systems' target audiences are those with advanced technical know-how on weather forecasting since the systems do not provide a conclusive description of weather condition. Instead, the user must deduce this from the illustrations of each parameter of the weather information. The complexity of interpreting the information from these systems makes the graphic representation impossible for laymen to interpret. 

\bigskip
Unlike these systems, the system that we are proposing is leaning towards developing a public visualization tool that is easy to interpret. The system will focus only on specific weather information - precipitation. Since tropical rain behaves differently from rains in other parts of the globe, some weather information may obscure the exact weather forecast in a tropical region.

\section{Visualizing Tropical Rain}
Two thirds of all the rain in the world falls in the tropical region \cite{adler}. Tropical rainfall is described as very uncharacteristic, as compared to other regions. In temperate climates, the rain cells are larger and the rain rate is relatively 'lighter' \cite{allnutt}. In stark contrast, rain in the tropics has relatively smaller rain cells that result to 'heavy' down pours over a short period. This uncharacteristic behavior is due to temperature changes in the atmosphere and the vast oceans that surround the tropics \cite{pids}. 

\bigskip
The different parts of the tropical region also experience varying difference in the duration, intensity and timing of rainy and dry periods \cite{pids}. One particular phenomenon that has been found to have caused the variability of rainfall intensity is the El Ni�o Southern Oscillation (ENSO). Climate analysts study strange occurrences, such as this, to understand tropical meteorology. 

\bigskip
Because of the uncharacteristic behavior of tropical rain, much research has been conducted to predict rain in the Philippines. Rainman, which is used in measuring rainfall and stream fall, is one of the tools that PAGASA uses to research on the predictability of rain \cite{pids}. This uses climate variables, such as the southern oscillation index and the sea surface temperature, measured over a one to twelve month period in three different locations to create forecasting models. The predictions from these models were verified with real climate data of the three sites over the past thirty years. Results from Rainman, as seen in Table \ref{table:rainman}, suggest that there is a higher chance of heavy rainfall in the Philippines during the Pacific Ocean's cool phase than its warm phase.

\begin{table}[h]
	\centering
		\caption{Probability of Rainfall During the Warm and Cool Phases of the Pacific Ocean}
		\label{table:rainman}
		\begin{tabular}{@{\extracolsep{\fill}} | p{.19\textwidth} | p{.3\textwidth}| p{.3\textwidth}| }
			\hline
			 \textbf{Study Site}& \textbf{During a cooler Pacific Ocean (Sep-Feb)} & \textbf{During a warmer Pacific Ocean (Mar-Sep)}\\
			\hline
			Malaybalay	& 70\%	& 20\% - 30\% \\
			\hline
			Tacloban	& 60\% - 80\%	& 20\% - 40\% \\
			\hline
			Tuguegarao &	60\% - 80\%	& Reduced	\\
			\hline
		\end{tabular}
\end{table}

\bigskip
However, very recent studies by UK and US researchers show otherwise. According to the research, there is an increase in the frequency of heavy rainfall when the tropics warmed up and then a decrease when it cooled down \cite{allan}. Studies suggest that this happens because during warm climate, the atmosphere can hold more water vapor \cite{allan} and there is more moisture over the ocean \cite{adler}.

\bigskip 
Aside from the studies that characterize or explain the behavior of rain, several researches have also been conducted regarding the visualization of rain. One of them was done at the University of Bonn in Denmark \cite{gerstner} using weather radars. The weather radars were used to provide rainfall rate estimates by measuring the backscatter or reflectivity from rain drops in the atmosphere. However, in their research, very large amount of data needs to be processed and this resulted in performance problems and delays in the delivery of the rain fall visualization. To resolve this, they used multiresolution algorithms which vary the data displayed relative to the scaling of the map. Zoomed maps are displayed in higher resolution and with more details while overview maps are just displayed in lower resolution. Although they were able to address this problem, their visualization tool still lacks display of backlog data that can be used for further research in the behavior of rain. Also, their visualization tool gets its data from weather radars, which, in as much as they are part of the new generation technologies, are expensive and not readily available here in the Philippines.

\bigskip
A similar research is done by Celano and his colleagues \cite{celano} in ARPA-SIM organization in Italy. Using satellites to retrieve the rain information and the Google Earth platform, they are able to display the accumulated rainfall and its distribution over the map. One of the features of their visualization tool is that it is able to determine whether there is light rain, moderate rain, heavy rain, rain hail mixture or snow. Another important feature is that their map is dynamic. It is able to ``play" the weather status depending on the time sequence specified. Furthermore, its real time visualization capability provides an effective early warning system for possible hazardous occurrences caused by very bad weather. One of its major limitations is its inaccuracy in the association of the location with its corresponding rain event. This is a result of its use of a geo-alignment system which gives a point in the map that is off by about 100 to 150 meters from the original location. Also, like the previous research mentioned, it only accepts data from its expensive satellites.

\section{Connecting to a Wireless Network}
	As opposed to wired communication devices, wireless technologies, such as PDAs, cell phones and laptops, connect to one or more wireless devices without network or peripheral cabling \cite{karygiannis}.  These devices use radio frequency transmissions to send and receive data. Wireless networks provide transport capabilities between and among wireless devices and the wired networks. Wireless networks are categorized into Wireless Wide Area Network (WWAN), Wireless Local Area Networks (WLAN), and Wireless Personal Area Networks (WPAN). 
	
\bigskip	
WLAN operates through a fixed network infrastructure \cite{karygiannis}. It is composed of client stations, such as wireless-enabled laptops and handheld devices. Client stations use wireless network interface cards (NIC) to communicate with each other via the access point (AP). The AP is similar to a cell-site base station in cellular communications. It is a stationary device that is ``a radio transceiver on one side and a bridge to the wired backbone on the other." 

\bigskip
WLAN offer more flexibility and portability over wired networks \cite{karygiannis}. However, it is subject to the limitations of radio frequency transmissions. Radio signals are propagated through reflection, diffraction or scattering \cite{ibe}. Radio wave is reflected when it is blocked by an object that is larger than its wavelength. Diffraction or bending of radio waves occurs when there is an obstruction to the radio path. Scattering of a radio wave happens when it encounters an object with smaller dimensions than its wavelength. Several factors that lead to the occurrence of the said radio propagation mechanisms contribute to signal propagation loss. This includes weather conditions, specifically, rain. 

\bigskip
In Wessling, Germany, researchers studied the correlation between rain intensity and signal attenuation of a 40 GHz Italsat beacon \cite{fiebig}. The results of the study showed that as rain intensity increases, the power of the wireless signal significantly decreases. 

\bigskip
In another study, it showed that rain attenuation cause the highest propagation impairment at frequencies above 10 GHz compared to other propagation factors, namely gaseous absorption, cloud attenuation, melting layer attenuation, tropospheric scintillations and low-angle fading \cite{dissanayake}. 

\bigskip
Some researchers utilized these findings to create models that predict rain attenuation base on rain rate. One particular study revealed that most of these models were accurate at medium to low rain rates in tropical regions. \cite{allnutt}

\bigskip
Although several studies have shown that there is a correlation between signal attenuation and rain intensity, most of this research was aimed at predicting rain attenuation based on rain intensity instead of the other way around. 

\bigskip
One particular research conducted by ECE majors in the Ateneo de Manila University utilized signal attenuation to measure rain intensity. The proponents used Wireless IP Access System (WIPAS) as part of their wireless sensing network \cite{abrajano}. They found a significant decrease in the signal attenuation (dB) for certain rain intensities. With this finding, they recommended the use of the wireless sensing device in a disaster detection and prevention system.

\section{Disaster Management System}

\subsection{Components of Disaster Management Systems}
Ideally, disaster management systems should support accurate disaster prediction, efficient information dissemination and well-organized damage assessment and rehabilitation \cite{singh}. Disaster prediction includes preventive measures, specifically, ``vulnerability analysis, hazard zonation and prior risk assessement, � timely and reliable weather forecasts and advance warnings of severe weather\cite{singh1}." Preparedness to disaster will significantly ``minimize the loss of life and damage and facilitate timely and effective rescue, relief and rehabilitation of the affected population \cite{singh1}".

\subsection{ICT in Disaster Management}
Since a disaster management system should efficiently mobilize people of different background and location, it must be capable of integrating information regardless of discipline, organization and geographic location \cite{venkatachary}. It should allow multiple accesses to valuable information depending on the level of access granted to the user. This disaster information can be further analyzed in order to pinpoint the right course of action that must be taken. Doing all of these manually is not only tedious but also inefficient, thus, technology must be utilized in a disaster management system.   

\bigskip
Technology can be used in ``data collection, networking, communication, warning dissemination, service delivery mechanisms, GIS databases, expert analysis systems, information resources etc. \cite{venkatachary1}" These functions can be performed by current technologies, which can be adopted into a disaster management system. This includes ``remote sensing, Global Positioning Systems (GPS), Data Collection Platforms (DCPs), hand-held GPS, Geographical Information System (GIS) \cite{venkatachary1}" and ``Geospatial models, Cyclone warning and Dissemination System (CWDS) \cite{venkatachary1}."

\bigskip
The concept of using information technology to make disaster management easier and more efficient is not new. However, there are only very few disaster management systems that exist today and only a few of them are widely deployed despite of their importance in disaster preparedness and mitigation \cite{careem}.

\bigskip
One of these projects made to provide efficient information management concerning disaster risk reduction is the Sahana Project \cite{careem}. Sahana has been deployed to manage major disasters all over the world; these include the 2004 Tsunami in Asia, 2005 Kashmir earthquake in Pakistan and the 2006 Guinsaugon landslides in the Philippines. Currently, it has modules such as missing person registry, request/aid management which coordinates specific aid requests with the available donations, and shelter registry which keeps track of all available evacuation centers and the facilities and services provided in those locations. Sahana, however, is considered only as a post-disaster management and humanitarian assistance tool - information are only collected and provided once a disaster has already occurred. At the moment, it lacks modules which can help in disaster preparedness and pre-disaster risk reduction.




