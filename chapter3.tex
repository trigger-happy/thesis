\chapter{Framework}

\section{Definition of Terms}

There are several terms that will be used throughout this paper.
The following terms however are central to the paper and should therefore be defined clearly
to avoid any misconceptions and misunderstandings:

\begin{itemize}
 \item Graphics Processing Unit (GPU) -  A Graphics Processing Unit or GPU is a special type
of computer device that deals with the processing of graphics that will be displayed
on screen. It is similar to a Central Processing Unit or CPU in their ability to execute
a sequence of instructions over a set of data. The difference however is that while the
CPU is capable of executing a wide range of instructions, the GPU is specialized to perform
highly intensive mathematical computations and over a large set of data in a parallel manner.

 \item Genetic Algorithm (GA) - Genetic Algorithms are part of a special family of algorithms
called Stochastic Search Algorithms, also known in simple terms as randomized search. GAs
follow nature's pattern of evolution and the concept of "survival of the fittest." Given
a random set of individuals, each one is evaluated and then scored according to a fitness function.
Individuals that obtain the best scores become part of the next generation of individuals while the
rest are replaced through reproduction (mixing elements of one individual with another individual).
Eventually, only the most "fit" of individuals will be part of the final generation.

 \item Fitness function - The fitness function is the part of the genetic algorithm that ranks
an individual according to a set of arbitrary criteria. It is the score that this function
gives to an individual that determines if the said individual will be part of the next generation
or not.

 \item Individual - An individual is a representation of a possible set of parameters to a problem.
These could be variables that may be plugged into an equation or the set of timing values for an
AI agent's reaction time. Individuals may conceptually be thought of as a possible solution to a
particular problem and the correctness of the solution is determined by the fitness function.

 \item Generation - A generation, also known as a population, is a set of individuals that represent
an iteration in the algorithm. Individuals in a generation are ranked by the fitness function and then
mixed with other individuals to generate a new set of individuals through a process akin to biological
reproduction.

  \item Optimizer - For this paper, it will refer to the application that will attempt to find an
optimal set of parameters (gene) for the game AI. Two versions of this will be developed, a CPU
based optimizer which only makes use of the main processor and the GPU version, which utilizes
the GPU for the bulk of its work.

  \item Physical Time - This refers to the time as perceived in the physical world
(clock time).

  \item Game Time - This may be accelerated to reduce the physical time for both
evolving the AI and observing the AI's performance in a game.

\end{itemize}

\section{Variables of the Study}

In order to measure the success of the project, there is a need to measure several variables. These
variables will determine if utilizing the GPU is not only efficient, but also acceptable in generating
an adequate solution to the AI problem.

\begin{itemize}
 \item Running time - The time it takes for an acceptable solution can be found in a single run.
This can pertain to the time it takes for either the CPU-based or GPU-based optimizer to find an
optimal solution to a particular problem.

 \item AI Performance - How well the AI performs in playing the game and achieving the intended goals
of the game. One of the criteria is the length of time that the AI player has remained alive in a single
game.
\end{itemize}