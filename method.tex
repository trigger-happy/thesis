\subsection*{Methodology}
Our first set of research questions involve the design of the application itself.
There are several things that must be done during the design phase. The first is
the design of the actual game where the AI must be placed in. For this we chose to
create a simple 2D tank game where the objective is to shoot the other tanks
while at at the same time avoiding death from the projectiles of other tanks. We
chose this kind of game as it is simple enough to create within a short amount of time
while at the same time being complex enough for the AI to perform a wide variety of
actions. Additional game elements like health points, player score, power ups and
obstacles may be implemented to increase or decrase the complexity of the game.


The next design element that needs to be tackled is the implementation of the AI itself.
Although we will be using Genetic Algorithms to evolve the AI in becoming optimal, it
will still require a framework which it will operate in. Such implementations include
Finite State Machines, Fuzzy State Machines, Hierarchical State Machines, Behavior Trees
and Hierarchical Task Networks. Each of these has its pros and cons and the choice of
which AI implementation to use will determine how the AI's "genetic" structure will be
designed and evolved.


Once we have chosen which AI implementation to use, the next design issue that needs to
be addressed is the way the "genes" should be structured for evolution. The simplest
structure would be a vector of numeric values that can be used to tune some specific
parameters in the AI's code. Things like the minimum amount of hp that the agent can have
before it attempts to run away from battle to heal or the minimum distance between the
agent and an incoming projectile before the agent should attempt to dodge. This structure
implies that the AI's decision making is already fixed and the only things that change
are the various parameters that will trigger a yes or no in any single decision. There
are other possible genetic structures, but these may be too complicated to implement
within a reasonable time frame.


Once the design decisions are in place, the next step is to create the simulation portion
of the game. This simulation code is essentially the heart of the game and will be used
both by the final game and the program which will evolve the game's AI.


The final step is to put the generated AI to the test. The AI agent of the game will make
use of the best parameters that have been evolved by the genetic algorithm and pit against
various opponents. These opponents include an AI with parameters that have been hand-tailored
through play testing and design and a human player. It is here that we determine how well
the evolved AI performs in general.